
\begin{thebibliography}{94}
%\cleardoublepage
%\phantomsection
\bibliographystyle{plain}
\bibitem[Currin et al.(1991)]{pd1} Currin, C., Mitchell, T.J., Morris, M.D., and Ylvisaker, D. (1991). Bayesian Prediction of Deterministic Functions, With Applications to the Design and Analysis of Computer Experiments. {\it Journal of the American Statistical Association}, $86(416), 953-963$. 

\bibitem[Martin \& Simpson(2004)]{pd2} Martin, J. D., \& Simpson, T. W. (2004). On the use of kriging models to approximate deterministic computer models. {\it In ASME 2004 International Design Engineering Technical Conferences and Computers and Information in Engineering Conference}, $481-492$.

\bibitem[Osio et al.(1996)]{pd3} Osio, I.G. and Amon, C.H. (1996). An Engineering Design Methodology with Multistage Bayesian Surrogate and Optimal Sampling. {\it Research in Engineering Design}, $8(4), 189-206$.

\bibitem[Sacks et al.(1989)]{pd4} Sacks, J., Welch, W., Mitchell, T., Wynn, H. (1998). Design and analysis of computer experiments. {\it Statistical Science}, $4(4), 409-435$.

\bibitem[Santner et al.(2003)]{pd5} Santner, T., Williams, B., Notz, W. (2003). The Design and Analysis of Computer Experiments. Springer.

\bibitem[Li \& Sudjianto(2005)]{pd6} Li, R., \& Sudjianto, A. (2005). Analysis of computer experiments using penalized likelihood in gaussian kriging models. {\it Technometrics}, $47(2), 111-120$.

\bibitem[Andrianakis \& Challenor(2009)]{pd7} Andrianakis, Y., \& Challenor, P. G. (2009). Parameter estimation and prediction using gaussian processes. {\it Technical report}, MUCM Technical report 09/05, University of Southampton.

\bibitem[Roustant et al.(2012)]{pd8} Roustant, O., Ginsbourger, D., \& Deville, Y. (2012). DiceKriging, DiceOptim: Two R packages for the analysis of computer experiments by kriging-based metamodeling and optimization.

\bibitem[Park \& Baek(2001)]{pd9} Park J.S., \& Baek, J. (2001). Efficient Computation of Maximum Likelihood Estimators in a
Spatial Linear Model with Power Exponential Covariogram. {\it Computer Geosciences}, $27, 1-7$.

\bibitem[Hankin(2005)]{pd10} Hankin, R. K. (2005). Introducing BACCO, an R package for Bayesian analysis of computer code output. {\it Journal of Statistical Software}, $14(16), 1-21$.

\bibitem[O'Hagan(2006)]{pd11} O'Hagan, A. (2006). Bayesian Analysis of Computer Code Outputs: A Tutorial. {\it Reliability Engineering and System Safety}, $91, 1290-1300$.

\bibitem[Conti et al.(2009)]{pd12} Conti, S., Gosling, J. P., Oakley, J. E., \& O'hagan, A. (2009). Gaussian process emulation of dynamic computer codes. {\it Biometrika}, asp028.

%\bibitem[Conti {\em et al}.(2004)]{pd13} Conti, S., Anderson, C. W., Kennedy, M. C., \& O’Hagan, A. (2004). A Bayesian analysis of complex dynamic computer models. {\it In Proc. of the 4th International Conference on Sensitivity Analysis of Model Output}.

\bibitem[Conti {\em et al}.(2010)]{pd14} Conti, S., \& O’Hagan, A. (2010). Bayesian emulation of complex multi-output and dynamic computer models. {\it Journal of statistical planning and inference}, $140(3), 640-651$.

\bibitem[Bhattacharya(2007)]{pd15} Bhattacharya, S. (2007). A simulation approach to Bayesian emulation of complex dynamic computer models. {i\t Bayesian Analysis}, $2(4), 783-815$.

%\bibitem[Kennedy {\em et al}.(2001)]{pd16} Kennedy, M.C., \& O'Hagan, A. (2001). Bayesian calibration of computer models. {\it Journal of the Royal Statistical Society}. Series B, Statistical Methodology, $425-464$.

\bibitem[Azman \& Kocijan (2005)]{pd17} Azman, K., \& Kocijan, J. (2005). Comprising prior knowledge in dynamic gaussian process models. {\it In Proceedings of the International Conference on Computer Systems and Technologies-CompSysTech}, Vol. 16(17.6).

\bibitem[Kleijnen(2009)]{pd18} Kleijnen, J. P. (2009). Kriging metamodeling in simulation: A review. {\it European Journal of Operational Research}, $192(3), 707-716$.

\bibitem[Kleijnen \& Simpson(2005)]{pd19} Martin, J. D., \& Simpson, T. W. (2005). Use of kriging models to approximate deterministic computer models. {\it AIAA journal}, $43(4), 853-863$.

\bibitem[Kleijnen \& Mehdad(2014)]{pd20} Kleijnen, J. P., \& Mehdad, E. (2014). Multivariate versus univariate kriging metamodels for multi-response simulation models. {\it European Journal of Operational Research}, $236(2), 573-582$.

\bibitem[Boukouvalas {\em et al}.(2009)]{pd21} Kleijnen, J. P. (2009) Boukouvalas, A., Cornford, D., \& Singer, A. (2009). Managing uncertainty in complex stochastic models: {\it Design and emulation of a rabies model. In 6th St. Petersburg Workshop on Simulation}, (pp. 839-841).

\bibitem[Kersting {\em et al}.(2007)]{pd22} Kersting, K., Plagemann, C., Pfaff, P., \& Burgard, W. (2007). Most likely heteroscedastic Gaussian process regression. {\it In Proceedings of the 24th international conference on Machine learning}, (pp. 393-400). ACM.

\bibitem[Kleijnen \& Beers(2005)]{pd23} Kleijnen, J.P., \& Van Beers, W.C. (2005). Robustness of kriging when interpolating in random simulation with heterogeneous variances: Some experiments. {\it European Journal of Operational Research}, $165(3), 826-834$.

\bibitem[Bates {\em et al}.(2006)]{pd24} Bates, R. A., Kenett, R. S., Steinberg, D. M., \& Wynn, H. P. (2006). Achieving robust design from computer simulations. {\it Quality Technology and Quantitative Management}, $3(2), 161-177$.

\bibitem[Bates {\em et al}.(1997)]{pd25} Goldberg, P. W., Williams, C. K., \& Bishop, C. M. (1997). Regression with input-dependent noise: A Gaussian process treatment. {\it Advances in neural information processing systems}, $10, 493-499$.

\bibitem[Henderson {\em et al}.(2012)]{pd26} Henderson, D. A., Boys, R. J., Krishnan, K. J., Lawless, C., \& Wilkinson, D. J. (2012). Bayesian emulation and calibration of a stochastic computer model of mitochondrial DNA deletions in substantia nigra neurons. {\it Journal of the American Statistical Association}.

\bibitem[Boukouvalas {\em et al}.(2014)]{pd27} Boukouvalas, A., Sykes, P., Cornford, D., \& Maruri-Aguilar, H. (2014). Bayesian precalibration of a large stochastic microsimulation model. {\it Intelligent Transportation Systems, IEEE Transactions on}, $15(3), 1337-1347$.

\bibitem[Kleijnen \& Mehdad(2012)]{pd28} Kleijnen, J., \& Mehdad, E. (2012). Kriging in multi-response simulation, including a Monte Carlo laboratory. {CentER Discussion Papers Series}, (2012-039).

\bibitem[Jarvis {\em et al}.(2005)]{l3} Jarvis, P., Jefferson, B., \& Parsons, S. A. (2005). Measuring floc structural characteristics. {\it Reviews in Environmental Science and Bio/Technology}, $4(1-2), 1-18$.

\bibitem[Frazer {\em et al}.(2013)]{l4}  Fraser, C. E., McIntyre, N., Jackson, B. M., \& Wheater, H. S. (2013). Upscaling hydrological processes and land management change impacts using a metamodeling procedure. {\it Water Resources Research}, $49(9), 5817-5833$.

 \bibitem[Wheater {\em et al}.(2008)]{l8} Wheater, H.S., B. Reynolds, N. McIntyre, M. Marshall, B. Jackson, Z. Frogbrook, I. Solloway, O. J. Francis, and J. Chell (2008). Impacts of upland land management on flood risk: Multi-scale modelling methodology and results from the Pontbren experiment, {\it FRMRC Res. Rep. UR}, 16, 163 pp., Imp. Coll. \& CEH Bangor, London, U.K.

 \bibitem[Van {\em et al}.(2009)]{l9} Van Oijen, M., Thomson, A., \& Ewert, F. (2009). Spatial upscaling of process-based vegetation models: An overview of common methods and a case-study for the UK. Methods, 1(3).
 
\bibitem[Ofiteru {\em et al}.(2014)]{l11} Ofiteru, I. D., Bellucci, M., Picioreanu, C., Lavric, V., \& Curtis, T. P. (2014). Multi-scale modelling of bioreactor–separator system for wastewater treatment with two-dimensional activated sludge floc dynamics. {\it Water research}, $50, 382-395$.

%\bibitem[Hankin(2005)]{pd32} Hankin, R. K. (2005). Introducing BACCO, an R package for Bayesian analysis of computer code output. {\it Journal of Statistical Software}, $14(16), 1-21$.

%%%%%%%%%%%%%%%%%




\bibitem[Higdon {\em et al}.(2008)]{q23} Higdon, D., Gattiker, J., Williams, B. \& Rightley, M. (2008). Computer model calibration using high-dimensional output. {\it Journal of the American Statistical Association}, $103, 570-583$.


\bibitem[Kennedy {\em et al}.(2001)]{45} Kennedy, M.C. \& O'Hagan, A. (2001). Bayesian calibration of computer models. {\it Journal of Royal Statistical Society}, series B, $63(3), 425-464$.

\bibitem[Kennedy {\em et al}.(2006)]{q17} Kennedy, M. C., Anderson, C. W., Conti, S., and O'Hagan, A. (2006). Case studies in Gaussian process modelling of computer codes. {\it Reliability Engineering \& System Safety}, $91, 1301-1309$.

%\bibitem[Kennedy {\em et al}.(2008)]{46} Kennedy, M.C. {\em et al}. (2008). Quantifying uncertainty in the biospheric carbon flux for England and Wales. {\it Journal of the Royal Statistical Society}, Series B, $171, 109-135$.




\bibitem[Oakley(1999)]{q35} Oakley, J. (1999). Bayesian Uncertainty Analysis For Complex Computer Codes. Ph.D. thesis, University of Sheffield.

\bibitem[Oakley \& O'Hagan(2002)]{60} Oakley, J. and O'Hagan, A. (2002). Bayesian inference for the uncertainty distribution of computer model outputs. {\it Biometrika}, $89, 769-784$.

\bibitem[Oakley \& O'Hagan(2004)]{q5} Oakley, J. E. and O'Hagan, A. (2004). Probabilistic sensitivity analysis of complex models: a Bayesian approach.{\it Journal of Royal Statistical Society}, $66B, 751-769$.


\bibitem[Oyebamiji {\em et al}.(2015)]{qwole} Oyebamiji, O.K., Edwards, N.R., Holden, P.B., Garthwaite, P.B., Schaphoff, S., and Gerten, D. (2015). Emulating global climate change impacts on crop yields. {\it Statistical Modelling}, 1471082X14568248.

%\bibitem[O'Hagan(2006)]{l5} O'Hagan, A. (2006). Bayesian Analysis of Computer Code Outputs: A Tutorial. {\it Reliability Engineering and System Safety}, $91, 1290-1300$.

\bibitem[Higdon {\em et al}.(2008)]{l6} Higdon, D., Gattiker, J., Williams, B. \& Rightley, M. (2008). Computer model calibration using high-dimensional output. {\it Journal of the American Statistical Association}, $103, 570-583$.

\bibitem[Quinonero-Candela \& Rasmussen(2005)]{q47} Quinonero-Candela, J., \& Rasmussen, C. E. (2005). A unifying view of sparse approximate Gaussian process regression. {\it The Journal of Machine Learning Research}, $6, 1939-1959$.


\bibitem[Rasmussen \& Williams(2006)]{q10} Rasmussen, C.E. and Williams, C.K.I. (2006). Gaussian Processes for Machine Learning, the MIT Press.


%\bibitem[Rougier(2007)]{68} Rougier, J.C. (2007). Probabilistic inference for future climate using an ensemble of climate model evaluations. {\it Climate Change}, $81, 247-264$.

%\bibitem[Rougier(2008)]{q28} Rougier, J. (2008). Efficient emulators for multivariate deterministic functions. {\it Journal of Computational and Graphical Statistics}, $17(4), 827-843$.

%\bibitem[Rougier {\em et al}.(2009)]{q29} Rougier, J., Guillas, S., Maute, A., \& Richmond, A. D. (2009). Expert knowledge and multivariate emulation: The thermosphere–ionosphere electrodynamics general circulation model (TIE-GCM). {\it Technometrics}, $51(4), 414-424$.

\bibitem[Sacks {\em et al}.(1989)]{q7} Sacks, J., Welch, W., Mitchell, T., Wynn, H. (1998). Design and analysis of computer experiments. {\it Statistical Science}, $4(4), 409-435$.

%\bibitem[Sacks {\em et al}.(2010)]{r8} Sacks,W.J., Deryng, D., Foley, J.A., Ramankutty, N. (2010). Crop planting dates: an analysis of global patterns. {\it Global Ecology and Biogeography}, $19, 607–620$.

\bibitem[Santner {\em et al}.(2003)]{70} Santner, T., Williams, B., Notz, W. (2003). The Design and Analysis of Computer Experiments. Springer.

%\bibitem[Shi {\em et al}.(2005)]{q18} Shi, J.Q., Murray-Smith, R. and Titterington, D.M. (2005). Hierarchical Gaussian Process mixtures for regression. {\it Statistics and Computing}, $15, 31-41$.

\bibitem[Wilkinson(2010)]{80} Wilkinson, R.D., in: Biegler {\em et al}. (Eds.) (2010). Large-Scale Inverse Problems and Quantification of Uncertainty. {\it John Wiley \& Sons}, New York.

\bibitem[Young {\em et al}.(2011)]{83} Young, P.C. and Ratto, M. (2011). Statistical emulation of large linear dynamic models. {\it Technometrics}, $53(1), 29-43$.
%%%%%%19-05-2016
\bibitem[Le Gratiet (2013)]{co1} Le Gratiet, L. (2013). Bayesian analysis of hierarchical multifidelity codes. SIAM/ASA {\it Journal on Uncertainty Quantification}, $1(1), 244-269$.

\bibitem[Le Gratiet \& Garnier(2014)]{co2} Le Gratiet, L., \& Garnier, J. (2014). Recursive co-kriging model for design of computer experiments with multiple levels of fidelity. {\it International Journal for Uncertainty Quantification}, $4(5).$

\bibitem[Forrester et al.(2007)]{co3} Forrester, A. I., Sobester, A., \& Keane, A. J. (2007). Multi-fidelity optimization via surrogate modelling. {\it In Proceedings of the royal society of london a: mathematical, physical and engineering sciences}, $463(2088), 3251-3269$. The Royal Society.

\bibitem[Kennedy \& O'Hagan(2000)]{co4} Kennedy, M. C., \& O'Hagan, A. (2000). Predicting the output from a complex computer code when fast approximations are available. {\it Biometrika}, $87(1), 1-13$.

\bibitem[Kuya et al.(2011)]{co5} Kuya, Y., Takeda, K., Zhang, X., \& J. Forrester, A. I. (2011). Multifidelity surrogate modeling of experimental and computational aerodynamic data sets. {\it AIAA journal}, $49(2), 289-298$.


\bibitem[Amaral et al.(1997)]{co6} Amaral, A. L., Alves, M. M., Mota, M., \& Ferreira, E. C. (1997). Morphological characterisation of microbial aggregates by image analysis. {\it Proceedings of the $9^th$ Pattern Recorgnition Conference}, $95-100$, Coimbra, 1997.

\bibitem[de Boer et al.(2000)]{co7} de Boer, D. H., Stone, M., \& Levesque, L. M. (2000). Fractal dimensions of individual flocs and floc populations in streams. {\it Hydrological Processes}, $14(4), 653-667$.

\bibitem[Zmeskal et al.(2001)]{co8} Zmeskal, O., Vesely, M., Nezadal, M., \& Buchnicek, M. (2001). Fractal analysis of image structures. {\it Harmonic and Fractal Image Analysis}, $3-5$.

\end{thebibliography}{}




