\documentclass[12pt,titlepage]{report}
\usepackage[authoryear,semicolon]{natbib}
\usepackage{pdfpages}
%\usepackage{chicago}
\usepackage[onehalfspacing]{setspace}
%\doublespacing
%\linespread{1.6}
\setstretch{1.5}
\usepackage{apalike}
\onehalfspacing
\usepackage[left=40mm,right=25mm,top=30mm,bottom=20mm,headsep=10mm]{geometry}
\geometry{a4paper}
\usepackage[titletoc]{appendix}
\usepackage{fnpara}
\usepackage{graphicx}
\usepackage{epstopdf}
%\usepackage{mathtools}
\usepackage{amsmath,amsfonts,amsthm,amssymb}
\usepackage{fancybox}

\usepackage[active]{srcltx}
\usepackage{afterpage}
\usepackage[mathscr]{eucal}
\DeclareGraphicsRule{.tif}{png}{.png}{`convert #1 `dirname
#1`/`basename #1 .tif`.png}
\usepackage{fancyhdr}
%\pagestyle{fancy}
%\fancyhf{}
%\rhead{ \leftmark}
%\rhead[\thepage]{\chaptermark}
%\lhead{Oyebamiji, O.K.}
%\cfoot{ \thepage}


\usepackage{setspace}
\usepackage[small]{caption}
\usepackage{color}
\usepackage{subcaption}
\usepackage{placeins}
\usepackage{epstopdf}
\usepackage{booktabs}
\usepackage{tikz}
\usepackage{subcaption}
%\usepackage{color,framed} %Utilisation des couleurs et de l'environnement shaded
%\definecolor{shadecolor2}{rgb}{.92,.92,.92} % choix de la teinte de ``shaded''
%\definecolor{shadecolor}{rgb}{.6,.95,.6} % choix de la teinte de ``shaded''
\usepackage[pagebackref=true,colorlinks=true, linkcolor=black,anchorcolor=blue,citecolor=black,filecolor=blue,menucolor=blue,
urlcolor=black,plainpages=false,pdfpagemode=UseThumbs,pdftitle={Titre},pdfauthor={oluwole},
pdfsubject={Thesis},pdfstartview=FitH]{hyperref} % Extensions PDF



%=========Standard sets===========:
\newcommand{\stsets}[1]{\mathbb{#1}}
\newcommand{\R}{\stsets{R}}
\newcommand{\N}{\stsets{N}}
\newcommand{\C}{\stsets{C}}

%===============Theorems etc.========================
%Paul
\newcommand{\bt}{{\bf t}}
\newcommand{\bB}{{\bf B}}
\newcommand{\bSigma}{{\bf \Sigma}}
\newcommand{\tbSigma}{{\tilde {\bf \Sigma}}}
\newcommand{\tbu}{{\tilde {\bf u}}}
\newcommand{\tbU}{{\tilde {\bf U}}}
\newcommand{\bh}{{\bf h}}
\newcommand{\bH}{{\bf H}}
\newcommand{\bD}{{\bf D}}
\newcommand{\bU}{{\bf U}}
\newcommand{\bu}{{\bf u}}
\newcommand{\bZ}{{\bf Z}}
\newcommand{\bz}{{\bf z}}
\newcommand{\bC}{{\bf C}}
\newcommand{\bc}{{\bf c}}
%%
\newcommand{\bx}{{\bf x}}
\newcommand{\bX}{{\bf X}}
\newcommand{\by}{{\bf y}}
\newcommand{\bY}{{\bf Y}}
\newcommand{\bE}{{\bf E}}
\newcommand{\bW}{{\bf W}}
\newcommand{\tbx}{{\tilde {\bf x}}}
\newcommand{\tbX}{{\tilde {\bf X}}}
\newcommand{\tby}{{\tilde {\bf y}}}
\newcommand{\tbY}{{\tilde {\bf Y}}}
\newcommand{\hbX}{{\hat {\bf X}}}
\newcommand{\hbY}{{\hat {\bf Y}}}
\newcommand{\hby}{{\hat {\bf y}}}
\newcommand{\ty}{{\tilde {y}}}
\newcommand{\hy}{{\hat {y}}}


\newcommand{\bLambda}{\mathbf{\Lambda}}
\newcommand{\bGamma}{\mathbf{\Gamma}}
\newcommand{\hbGamma}{\hat {\mathbf{\Gamma}}}

\newcommand{\bgamma}{{\boldsymbol{\gamma}}}
\newcommand{\bepsilon}{{\boldsymbol{\varepsilon}}}
\newcommand{\tbepsilon}{{\tilde{\boldsymbol{\varepsilon}}}}
\newcommand{\hbepsilon}{{\hat{\boldsymbol{\varepsilon}}}}
\newcommand{\bbeta}{{\boldsymbol{\beta}}}
\newcommand{\hbbeta}{{\hat{\boldsymbol{\beta}}}}

\theoremstyle{definition}
\newtheorem{definition}{Definition}
\theoremstyle{remark}
\newtheorem{remark}[definition]{Remark}
\newtheorem{example}[definition]{Example}

%\newtheoremstyle{mytheorem}{0.5cm}{0.2cm}{\slshape}{ }{\bfseries}{.}{}{}
%\theoremstyle{my theorem}
\newtheorem{Th}[definition]{Theorem}
%\newtheorem{Prop}[definition]{Proposition}
\newtheorem{lemma}[definition]{Lemma}
\newtheorem{Cor}[definition]{Corollary}
%\restylefloat{figure}

%==========Probabilistic===========:
%\renewcommand{\cite}{\shortciteN}
\renewcommand{\P}{\mathbf{P}}
\renewcommand{\Re}{\mathrm{Re\,}}
\renewcommand{\Im}{\mathrm{Im\,}}
%\newcommand{\Prob}[1]{\mathbf{P}\{#1\}}
\DeclareMathOperator{\E}{{\bf E}}
\DeclareMathOperator{\var}{{\bfvar}}
\DeclareMathOperator{\supp}{supp}
\DeclareMathOperator{\dist}{dist}
\DeclareMathOperator{\one}{{1\hspace*{-0.55ex}I}}
%\DeclareMathOperator{\one}{{\mathbf{1}}}
\newcommand{\Ind}[1]{\one_{#1}}
\newcommand{\cond}{\hspace*{1ex} \rule[-1ex]{0.15ex}{3ex}\hspace*{1ex}}

%==========Divers===========:
\DeclareMathOperator*{\ssum}{{\textstyle \sum}}
%small\sum for\sum\delta_x
\newcommand{\comp}{\mathbf{c}}
\newcommand{\mydot}{{\raisebox{.3ex}{$\scriptscriptstyle{\,\bullet\,}$}}}
%\newcommand{\myline}{\newline\underline{\hskip\textwidth}}
\newcommand{\mytimes}{\!\times\!}
\newcommand{\mytilde}{{\!\raisebox{-0.9ex}{$\tilde{\ }$}}}
\renewcommand{\epsilon}{\varepsilon}
\renewcommand{\vec}[1]{\mathbf{#1}}
%\renewcommand{\phi}{\varphi}
\newcommand{\ti}{\to\infty}
\newcommand{\ssp}{\hspace{2pt}}
\newcommand{\seg}{see, \hbox{e.\ssp g.,}\ }
\newcommand{\ie}{\hbox{i.\ssp e.}\ }
\newcommand{\eg}{\hbox{e.\ssp g.,}\ }
\newcommand{\cf}{\hbox{c.\ssp f.}\ }
\newcommand{\etc}{\hbox{etc.}\ }
\newcommand{\iid}{\hbox{i.\ssp i.\ssp d.}\ }
\newcommand{\as}{\hbox{a.\ssp s.s}\ }
\newcommand{\viz}{\hbox{viz.}\ }
\renewcommand{\bibname}{References}
\renewcommand{\abstractname}{Abstract}


%\pagenumbering{roman}

\begin{document}
\chapter*{Emulation of Lammp outputs}
The goal of this report is to describe the procedure for building a simple emulator of major Lammp outputs.
\section*{Lammp emulator}
\subsection*{Experimental design}
This section describes the procedure for generating the parameter combinations and variables at which the Lammp model is run. We run the Lammp code for a small sample of inputs using a Latin Hypercube Design (LHD). This produces data for training the Lammp emulator to approximate the major Lammp outputs. LHD provides a good coverage of the input space with relatively small number of design points. We use maximin LHS technique that optimises samples by maximizing the minimum distance between design points. Suppose we want to sample a function of $p$ variables, the range of each variable is divided into $n$ probable intervals, $n$ sample points are then drawn such that a Latin Hypercube is created.

We generate an $n \times p$ variables Latin Hypercube sample matrix with values uniformly distributed on interval [0,1]. We then transformed the generated sample matrix to the quantile of a uniform distribution using the range of the parameters given in Table \ref{mytab1}. We limit our initial analysis to just $n=100$ training points for each dimension of the $p=22$ input spaces that are to be varied because Lammp model are computationally demanding. %The inversion of the correlation matrix under GP regression becomes more difficult as we increase the number of training points. 

%(in this report, we consider only mean floc diameter)

\subsection*{Simulation data}
We describe 2 different simulation procedures. Firstly, let the design matrix which contain the input to the Lammp model be denoted by $DM=(\theta^i_p, t_p, p=1,\ldots,22; i=1,\ldots,100)$, where the subscript $p$ represents the 22 input parameters and superscript $i$ denote 100 different realisations (design points), $t_p$ is the time in seconds at which the output data is recorded. The design matrix $D_{100 \times 22}$ denotes the input values at which the Lammp model will be run for every combinations of $x_p$ where $x_p$ represents $p^{th}$ row of $D$. The current Lammp code is set up to produce the following outputs namely particle diameter, position (3-dimensional), velocity (3-dimensional) and force (3-dimensional). The lammp code could be run for as long as there are sufficient computing resources to store the outputs. For the present analysis, the code is run for 352800 seconds to generate sufficient data for the emulation which is equivalent to $\approx$ 4 days of real time and generated output results are saved at a time-step of 2000 seconds which gives about 176 different time slices. 

Therefore, a single run of the code for each of design point $x_p$ will produce 10 different outputs $Y(x)= [y_1,\ldots,y_{10}]$ and 176 time steps. There are 5 different types of particle in the simulation namely AOB, NOB, HET, EPS and inert. We note that the shape and number of particles as well as the composition of the floc at each time step vary in the simulation. 

We perform the second simulation using the same input configuration as described above but repeated the runs for 5 times in order to incorporate stochastic variations in our outputs. This of course increase the amount of CPU time for the entire simulations even with running of codes in parallel.%with 5 repetitions


\section*{Methods}
\subsection{Procedure}
The work focuses on predictions of mean floc diameter from Lammp simulation outputs together with their associated uncertainty levels. We proposed a two-stage approach where we combined a linear model in the first stage and a Gaussian process regression for residual interpolation in the second stage.

Here, we describe the procedure for emulating the particle floc diameter. We start by computing the average diameter of all the particles at each time step to obtain a vector $y(x)=\bar z_1,\ldots,\bar z_{176}$ of the mean floc diameter to be emulated such that
\begin{equation}
\bar z_t=\frac{\sum^N_{k=1} z_{kt}{N}
\end{equation}
where $k=1,\ldots, N$, $N$ represents the total number of all 5 particles at each time slice and $z_{kt}$ is a simulation particle $k$ at time $t$.
For the training data, we sub-sample 75 out of 176 time slices for each design point $x_p$ which gives a total of 7500 observations.
The training data are $7500 \times 1$ vector $Y$ of mean floc diameter of Lammp output and $7500 \times 23$ matrix $X$ of input parameters (NOTE: The sequence of time $t$ at which output is sought is used as an additional input to the emulator). We fit a linear model to data ${Y,X}$ and use the fitted model to predict the left out observations for cross validation purpose.

We then obtain both the predictions and standard error of predictions of linear which is a measure of uncertainty in the predictions. Finally, we apply GP to model the residual from linear model.

\subsection*{GP modelling of residual data}


\begin{table}
\caption{List of all the parameters}\label{mytab1}
\centering
\fbox{
\begin{tabular}{*{2}{c}}
Parameters& Value\\ 
\hline
variable KsHET &0.01\\
variable Ko2HET & 0.81\\
variable Kno2HET & 0.0003\\
variable Kno3HET & 0.0003\\

variable Knh4AOB & 0.001\\
variable Ko2AOB & 0.0005\\

variable Kno2NOB & 0.0013\\
variable Ko2NOB & 0.00068\\


%#Defining maximum growth variables: list from Jaya 
variable MumHET & 0.00006944444\\
variable MumAOB & 0.00003472222\\
variable MumNOB & 0.00003472222\\
variable etaHET & 0.6\\

%#Defining decay rates variables: list from Jaya 
variable bHET & 0.00000462962\\
%#variable bHET & 0.00462962\\
variable bAOB & 0.00000127314\\
%#variable bAOB & 0.00127314\\
variable bNOB & 0.00000127314\\
%#variable bNOB & 0.00127314\\
variable bEPS & 0.00000196759\\

variable YEPS & 0.18\\
variable YHET & 0.61\\
variable EPSdens & 30\\
variable EPSratio & 1.25\\

variable factor & 1.5\\
variable ke & 5e+10
\end{tabular}}
\end{table}


These are general steps taken for emulating mean floc diameter and variance:
\begin{itemize}
\item[{(i)}] Design of experiment to identify questions to answer. 
\item[{(ii)}] Determine the relevant input factors to be varied (use only 22 for now). 
\item[{(iii)}] Assign uniform probability density to each input variable (little knowledge about the parameter). 
\item[{(iv)}] Generate 100 samples from each input distribution. 
\item[{(v)}] Evaluate the Lammp model for the various input combinations to obtain the training data.
\item[{(vi)}] Fit a linear model to the data using all 23 parameters (time included as additional variables)
\item[{(vii)}] Perform the GP emulation on the unexplained residual data
\end{itemize}

To construct our emulator for the prediction of floc diameter, we model the Lammp simulation output by fitting a linear model to the data in the first stage and apply a Gaussian process regression for residual analysis
%\begin{equation}\label{geneq}
%y=f(X) = \beta_0+\beta_{1}x_{1}+\ldots +\beta_{p}x_{p}+ \beta_{1}^{2}x_{1}^{2}+\ldots+\beta_{p}^{2}x_{p}^2+\beta_{1}\beta_{2}x_{1}x_{2}+\dots +\beta_{(p-1)}\beta_{p}x_{(p-1)}x_{p} 
%\end{equation}
\begin{equation}\label{geneq}
\by=f(\textbf{X})+\boldsymbol\varepsilon= \boldsymbol\beta_0+\boldsymbol\beta_{1}x_{1}+\ldots +\boldsymbol\beta_{p}x_{p}+\boldsymbol\varepsilon
% \boldsymbol\beta_{1}x_{1,1}^{2}+\ldots+\boldsymbol\beta_{p}x_{p,p}^2+\boldsymbol\beta_{1,2}x_{1}x_{2}+\dots +\boldsymbol\beta_{(p-1),p}x_{(p-1)}x_{p}
\end{equation}
where $\by$ is the Lammp simulated mean floc diameter, $p$ is the number of parameters for estimation and $x_1,\ldots,x_p$ are independent variables. We assume $\boldsymbol \varepsilon \sim N(0, \sigma^2)$. 












\end{document}